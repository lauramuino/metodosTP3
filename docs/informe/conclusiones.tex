\section{Conclusiones}

En este trabajo pudimos ver una aplicación real para los métodos de interpolación más comunes, lineal y cúbico. Analizamos el comportamiento de cada método para una serie de videos específicos, y pudimos comprobar que, aunque el PSNR da menor en el método de splines, esto no significa necesariamente que el video pierda fluidez, lo que si pasa en el método del vecino más cercano.\\

A pesar de que nosotros vimos los errores (artifacts) que se produjeron en cuadros específicos, esto no es tan notorio en el video que se genera, aunque si se agregan demasiados frames, el video comienza a verse borroso y molesto a la vista. También vimos que en algunos casos los artifacts se pueden arreglar haciendo alguna pequeña modificación en el algoritmo, como en el caso de los artifacts producidos por los cambios de cámara.\\

Un aspecto que nos pareció interesante, fue el hecho de que el uso de estos algoritmos para la generación de videos en cámara lenta, puede ahorrar dinero en la compra de cámaras que posean dicha funcionalidad incorporada, y puede también ahorrar tiempo de transferencia entre la cámara y el centro de cómputos. Además, vimos que la complejidad temporal es lineal con respecto a la cantidad de frames.