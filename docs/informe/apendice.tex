\section{Apéndice A: Enunciado}

%%%%%%%%%%%%%%%%%%%%%%%%%%%%%%%%%%%%%%%%%%%%%%%%%%%%%%%%%%%%%%%%%%%%%

\section{Apéndice B}
%En  el  apendice  B  se  incluiran  los codigos fuente de las funciones relevantes desde el punto de vista numerico.  Resultados que  valga  la  pena  mencionar  en  el  trabajo  pero  que  sean  demasiado  especificos  para aparecer en el cuerpo principal del trabajo podran mencionarse en sucesivos apendices rotulados con las letras mayusculas del alfabeto romano.  Por ejemplo:  la demostracion de una propiedad que aplican para optimizar el algoritmo que programaron para resolver un problema.

\section{Referencias}

\begin{itemize}
\item Bryan, Leise - 2006 - The Linear Algebra behind Google.
\item  Govan, Meyer, Albright - 2008 - Generalizing Google’s PageRank to Rank National Football League Teams.
\item Kamvar, Haveliwala - 2003 - Extrapolation methods…omputations.
\item https://atlas.mat.ub.edu/personals/dandrea/2012_09_25_escrito_google.pdf
\item CSR: http://netlib.org/linalg/html_templates/node91.html
\end{itemize}